\documentclass[11pt]{article}
\usepackage[super,square,comma]{natbib}
\usepackage[letterpaper, margin = 1.in]{geometry}
\usepackage{tikz}
\usepackage{lipsum}
\usepackage{setspace}
\usepackage{enumerate}
\usepackage[inline, shortlabels]{enumitem}
\usepackage{amsmath}
\newtheorem{problem}{Problem}
\numberwithin{equation}{section}

\usepackage{animate}
\usepackage{siunitx}
\usepackage{csquotes}
\usepackage{graphicx}
\usepackage{authblk}
\usepackage[labelfont=bf]{caption}
\usepackage{subcaption}

\usepackage{fancyhdr}
\pagestyle{fancy}
\lhead{Team 93463}
% \chead{\textsc{Multi-hop, HF Radio Propagation Near Japan}}
\rhead{Page \thepage\ of \pageref{LastPage}}
\setlength{\headheight}{14pt}

\usepackage{lastpage}

\renewcommand{\thefootnote}{\arabic{footnote}}
\newcommand{\cn}{$^{\text{[citation needed]}}$} %citation needed

% %%% Title Page Author Stuff %%%
% \renewcommand*{\Authsep}{, }
% \renewcommand*{\Authand}{, }
% \renewcommand*{\Authands}{, }
% \renewcommand*{\Affilfont}{\normalsize\normalfont}
% % \renewcommand*{\Authfont}{\bfseries}    % make author names boldface    
% \setlength{\affilsep}{1em}   % set the space between author and affiliation

% \newsavebox\affbox

% \author[1]{Colin M. Adams}
% \author[1]{Rachel L. Barcklay}
% \author[1,2]{Carla J. Becker}
% \affil[1]{%
%   \savebox\affbox{\Affilfont Department of Physics, Harvey Mudd College}%
%   \parbox[t]{\wd\affbox}{\protect\centering Department of Physics, Harvey Mudd College}} 
% \affil[2]{Department of Chemistry, Harvey Mudd College \par Claremont, CA 91711}



%%%%%%%%%%%%%%%%%%%%%%%
\renewcommand{\baselinestretch}{2}
\newcommand{\shrug}[1][]{%
\begin{tikzpicture}[baseline,x=0.8\ht\strutbox,y=0.8\ht\strutbox,line width=0.125ex,#1]
\def\arm{(-2.5,0.95) to (-2,0.95) (-1.9,1) to (-1.5,0) (-1.35,0) to (-0.8,0)};
\draw \arm;
\draw[xscale=-1] \arm;
\def\headpart{(0.6,0) arc[start angle=-40, end angle=40,x radius=0.6,y radius=0.8]};
\draw \headpart;
\draw[xscale=-1] \headpart;
\def\eye{(-0.075,0.15) .. controls (0.02,0) .. (0.075,-0.15)};
\draw[shift={(-0.3,0.8)}] \eye;
\draw[shift={(0,0.85)}] \eye;
% draw mouth
\draw (-0.1,0.2) to [out=15,in=-100] (0.4,0.95); 
\end{tikzpicture}}


\title{
    \textsc{{Multi-hop, High-Frequency Radio Propagation Near Japan}}
    }

\author{\Large Team 93463}
\date{\empty}
        

\begin{document}
%%%%%%%%%%%%%%%%%%%%%%%%%%%%%%%%%%%%%%%%%%%%%%%%%%%%%%%%%%%%%%%%%%%%%%
% Title Page
%%%%%%%%%%%%%%%%%%%%%%%%%%%%%%%%%%%%%%%%%%%%%%%%%%%%%%%%%%%%%%%%%%%%%%
    \singlespacing %sets spacing of abstract to single
    \clearpage 
    \maketitle
    \thispagestyle{fancy} %gets rid of page number at bottom 

    \noindent To whom it may concern, \vspace{4pt}\\
    %
    \indent When reading our solution, we hope that you are can read it using Adobe Acrobat Reader. This is because we have \texttt{.gif} files that can only be played with that particular PDF reader. Please visit \texttt{https://get.adobe.com/reader/} to download it if necessary. We are trying to put our best foot forward and we would appreciate if you receive our work as intended. We believe it is worth it. Thank you.
    \vspace{4pt} \\ 
    Respectfully,\vspace{3pt}\\
    \indent Team 93463
    \begin{center}
        \rule{12cm}{0.4pt}  
    \end{center} 

\begin{abstract}

    Despite being one of the main forms of communication worldwide for over a century, there are many unresolved questions from a fundamentally scientific point of view. This is problematic because, by not understanding the fundamental science of radio-propagation, peoples lives can be put at risk if limitations in radio-technology are not properly understood. We can imagine sailors at sea with nothing but a 100\,\si{\watt} high-frequency radio who are caught in the middle of a storming, and turbulent ocean. Because high-frequency waves propagate by reflecting off of the ionosphere (which in turn is affected by the time of day, time of year, the Sun's solar activity at the time, etc.) and the ocean, depending on the particular conditions, the radio the sailors have could be entirely inadequate to be of any use to them. 

    With this in mind, we attempted to model high frequency radio-waves and their interactions with the ionosphere, turbulent and calm oceans, and smooth and rugged terrain. To do this, we used used a simple and isotropic model for the ionosphere, modifying the Chapman Law (which quantifies the charge density as a function of altitude) to make it time-dependent. In other words, we made the charge density of the ionosphere dependent on the time of day and the time of year. To model the ocean, we had a simple model in which ocean waves have an  amplitude and wavelength that can be varied. By doing this, we are able to model the ocean's ability to transmit signal with integrity. We quantify this with a parameter that we call a ``quasi-index of refraction.''

<<<<<<< HEAD
    Using this, that a 100\,\si{\watt} HF signal form Tokyo to Los Angeles can reflect off of the ionosphere and ocean eight times each before signal integrity is lost. This approach can be applied to terrestrial mediums as well, namely mountains and other rugged terrains, as well as be applied to boats traveling in turbulent waters.

=======
    Using this we found after 6 hops the signal to noise ratio drops below 10 dB and after the first reflection from a turbulent ocean the decibel level was 64.2 dB from 779 dB. We were not able to calculate a different decibel level for after the first reflection from a calm ocean.
>>>>>>> 7cb57d488eedf007b3638b039d130516febee740
\end{abstract}

  \begin{center}
        \rule{12cm}{0.4pt}  
    \end{center}
\newpage


\end{document}
