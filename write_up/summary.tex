\documentclass[11pt]{article}
\usepackage[super,square,comma]{natbib}
\usepackage[letterpaper, margin = 1.25in]{geometry}
\usepackage{tikz}
\usepackage{lipsum}
\usepackage{setspace}
\usepackage{enumerate}
\usepackage[inline, shortlabels]{enumitem}
\usepackage{amsmath}
\newtheorem{problem}{Problem}
\numberwithin{equation}{section}

\usepackage{animate}
\usepackage{siunitx}
\usepackage{csquotes}
\usepackage{graphicx}
\usepackage{authblk}
\usepackage[labelfont=bf]{caption}
\usepackage{subcaption}

\usepackage{fancyhdr}
\pagestyle{fancy}
\lhead{Team 93463}
% \chead{\textsc{Multi-hop, HF Radio Propagation Near Japan}}
\rhead{Page \thepage\ of \pageref{LastPage}}
\setlength{\headheight}{14pt}

\usepackage{lastpage}

\renewcommand{\thefootnote}{\arabic{footnote}}
\newcommand{\cn}{$^{\text{[citation needed]}}$} %citation needed

% %%% Title Page Author Stuff %%%
% \renewcommand*{\Authsep}{, }
% \renewcommand*{\Authand}{, }
% \renewcommand*{\Authands}{, }
% \renewcommand*{\Affilfont}{\normalsize\normalfont}
% % \renewcommand*{\Authfont}{\bfseries}    % make author names boldface    
% \setlength{\affilsep}{1em}   % set the space between author and affiliation

% \newsavebox\affbox

% \author[1]{Colin M. Adams}
% \author[1]{Rachel L. Barcklay}
% \author[1,2]{Carla J. Becker}
% \affil[1]{%
%   \savebox\affbox{\Affilfont Department of Physics, Harvey Mudd College}%
%   \parbox[t]{\wd\affbox}{\protect\centering Department of Physics, Harvey Mudd College}} 
% \affil[2]{Department of Chemistry, Harvey Mudd College \par Claremont, CA 91711}



%%%%%%%%%%%%%%%%%%%%%%%
\renewcommand{\baselinestretch}{2}
\newcommand{\shrug}[1][]{%
\begin{tikzpicture}[baseline,x=0.8\ht\strutbox,y=0.8\ht\strutbox,line width=0.125ex,#1]
\def\arm{(-2.5,0.95) to (-2,0.95) (-1.9,1) to (-1.5,0) (-1.35,0) to (-0.8,0)};
\draw \arm;
\draw[xscale=-1] \arm;
\def\headpart{(0.6,0) arc[start angle=-40, end angle=40,x radius=0.6,y radius=0.8]};
\draw \headpart;
\draw[xscale=-1] \headpart;
\def\eye{(-0.075,0.15) .. controls (0.02,0) .. (0.075,-0.15)};
\draw[shift={(-0.3,0.8)}] \eye;
\draw[shift={(0,0.85)}] \eye;
% draw mouth
\draw (-0.1,0.2) to [out=15,in=-100] (0.4,0.95); 
\end{tikzpicture}}


\title{
    \textsc{{Multi-hop, High-Frequency Radio Propagation Near Japan}}
    }

\author{\Large Team 93463}
\date{\today}
        

\begin{document}
%%%%%%%%%%%%%%%%%%%%%%%%%%%%%%%%%%%%%%%%%%%%%%%%%%%%%%%%%%%%%%%%%%%%%%
% Title Page
%%%%%%%%%%%%%%%%%%%%%%%%%%%%%%%%%%%%%%%%%%%%%%%%%%%%%%%%%%%%%%%%%%%%%%
    \singlespacing %sets spacing of abstract to single
    \clearpage 
    \maketitle
    \thispagestyle{fancy} %gets rid of page number at bottom 

    \noindent To whom it may concern, \vspace{4pt}\\
    %
    \indent When reading our solution, we hope that you are can read our document in Adobe Acrobat Reader as we have \texttt{.gif} files that can only be played with that particular PDF reader. Please visit \texttt{https://get.adobe.com/reader/} to download it if need be. 

    We are trying to put our best foot forward and we would appreciate if you receive our work as intended. It is worth it. We promise. Thank you.
    \vspace{4pt} \\ 
    Respectfully,\\
    Team 93463
    \begin{center}
        \rule{12cm}{0.4pt}  
    \end{center}

\begin{abstract}
    We attempted to model high frequency radio-waves and their interactions with the ionosphere, turbulent and calm oceans, and smooth and rugged terrain. To do this, we used used a simple and isotropic model for the ionosphere, modifying the Chapman Law to make it time-dependent, i.e. dependent on the time of day and the time of year. To model the ocean, we had a simple model in which ocean wave amplitude and wavelength are varied. By doing this we are able to change its index of refraction.

    Using this we found...
    \\\\
    Hi team, here is what we \emph{need} to have in our report (according to the MCM overlords):
\begin{itemize}
    \item Restatement and clarification of the problem
    \item Explain assumptions and rationale/justification
    \item Include your model design and justification
    \item Describe model testing and sensitivity analysis
    \item Discuss the strengths and weaknesses
\end{itemize}

They also claim to judge the quality of our writing. So remember our good friend Williams. \shrug
\end{abstract}
  \begin{center}
        \rule{12cm}{0.4pt}  
    \end{center}
\newpage


\end{document}
